  % UTF-8 encoding
\documentclass[9pt, dvipsnames]{beamer} %
% Beamer 设置
\usetheme[secheader]{Boadilla} % 使用的 Beamer 主题: Boadilla
\usecolortheme{beaver} % 使用的 Beamer 颜色:beaver
% 字体设置
\usefonttheme{professionalfonts} % professional 字体
% 其他 Package
\usepackage{times}
\usepackage{amsmath}
\usepackage{verbatim}
\usepackage{anyfontsize}
\usepackage{subcaption} % 子图片
\usepackage{graphicx} % 图片
\usepackage[export]{adjustbox}
\setbeamertemplate{caption}[numbered]
\newcounter{saveenumi}
\resetcounteronoverlays{saveenumi}
\usepackage[multidot]{grffile} % 允许文件名带多个点
\usepackage{tabularx} % 表格
\usepackage{tikz}
\usepackage{ctex}
\usepackage{multicol}
%%%%%%%%%%%%%%%%%%%

%\usepackage{ctex} % xelatex 中文

\title{Example Code of Beamer} % 标题
\author[Your Name]{Your Name} % 作者
\date{} % 如果 Date 参数为空,自动显示当前日期

\begin{document}

\everymath{\displaystyle}

    % 标题页
\begin{frame}
    \titlepage % 根据上面信息生成标题
\end{frame}

\begin{frame}
    \frametitle{\textbf{Index}}
    \begin{multicols}{2}
    \tableofcontents
    \end{multicols}
\end{frame}



\section{Aspects of a novel nonlinear electrodynamics in flat spacetime and in a gravity-coupled scenario}




\subsection{非线性电动力学模型和场方程}

\begin{frame}

麦克斯韦的拉氏量

$$
F
=\frac{1 }{2 } (B^2-E^2)
$$

$$
L_{\mathrm{maxwell}}
=-F
$$

BI 模型的拉氏量

$$
L_{\mathrm{BI}}
=\frac{2 }{\beta } \left(1-\sqrt{1+\beta F} \right) 
$$

$\beta $ 是任意常数。

弱场极限 $\beta F\ll 1 $ 下,BI 模型的拉氏量可近似为:

$$
L_{\mathrm{BI}}
=\frac{2 }{\beta } \left(1-\sqrt{1+\beta F} \right)
\approx -F + \frac{1 }{4 } \beta F^2 - \frac{1 }{8 } \beta^2 F^3 +\mathcal{O}\left(\beta^3 F^4 \right) 
$$

当 $\beta\to 0 $,BI 模型的拉氏量与线性麦克斯韦的拉氏量相同。

\end{frame}

\begin{frame}
novel NLE 模型的拉氏量

$$
L_{\mathrm{general}}(F)
=-\frac{\left(aF+1 \right)^m }{\delta(bF+1)^n } \left(\beta F \right)^p
$$

$m,n,p $ 是无量纲常数,$a,b,\beta,\delta $ 是长度平方量纲的任意常数。

在弱场极限下,拉氏量可近似为:

$$
L_{\mathrm{general}}(F)
=-\frac{\left(aF+1 \right)^m }{\delta(bF+1)^n } \left(\beta F \right)^p
\approx -c\left[F^p + c_1 F^{p+1} +c_2 F^{p+2}  + \mathcal{O}\left(c_3 F^{p+3} \right) \right]
$$

$p=1 $ 时得到麦克斯韦的拉氏量。

通过分析取 $m=1,n=m+1,a=-3b $

得到含有两个参数且遵守麦克斯韦极限的拉氏量:

$$
L(F)
=\frac{\gamma(3\eta F - 1 )F }{(1+\eta F)^2 }
$$

其中,$\gamma=\beta/\delta $ 和 $\eta $ 是任意参数。

当 $\eta F\ll 1 $,即弱场极限下,拉氏量近似为:

$$
L(F)
=\frac{\gamma(3\eta F - 1 )F }{(1+\eta F)^2 } 
\approx -\gamma F + 5\gamma \eta F^2 -9\gamma \eta^2 F^3 + \gamma\mathcal{O}\left(\eta^3F^4 \right) 
$$
\end{frame}

\begin{frame}
利用电位移矢量 $\vec{D} $ 与 $\vec{E} $ 的关系 $\vec{D}=\partial L/\partial \vec{E} $,可由拉氏量式得到:

$$
\vec{D}
=\gamma\frac{1-7\eta F }{(1+\eta F)^3 } \vec{E}
$$

(10) $D_i = \varepsilon_i^{~~ j } E_j $

$$
\varepsilon_{ij} = \gamma \frac{1-7\eta F }{(1+\eta F)^3 }\delta_{ij} 
$$

磁场 $\vec{H}=-\partial L/\partial \vec{B} $ 结合拉氏量有

$$
\vec{H}
=\gamma \frac{1-7\eta F }{(1+\eta F)^3 } \vec{B}
$$

磁感应强度 $B_i=\mu_i^{~~j}H_j $

磁导率张量的逆 $\left(\mu^{-1} \right)_{ij} $

$$
\left(\mu^{-1} \right)_{ij}
=\gamma \frac{1-7\eta F }{(1+\eta F)^3 } \delta_{ij}
$$

可以认为新 NLE 拉氏量由这种特殊的介质生成。
\end{frame}

\begin{frame}
    平坦时空中拉氏量给出 E-L 运动方程:

    $$
    \partial_\mu\left(L_F F^{\mu\nu} \right) = 0
    $$
    
    其中,
    
    $$
    L_F
    \equiv \frac{\partial L }{\partial F } 
    =\frac{\gamma(-1+7\eta F) }{(1+\eta F)^3 }
    $$
    
    $F^{\mu \nu} $ 是麦克斯韦场强张量。

    可以回到无源麦克斯韦方程:

    $$
    \nabla\cdot\vec{D} = 0,\quad
    \frac{\partial \vec{D} }{\partial t } - \nabla\times\vec{H}= \vec{0} 
    $$
    
    由 Bianchi identity $\partial_\mu \tilde{F}^{\mu \nu}=0 $,$\tilde{F}^{\mu\nu} $ 是场强张量的对偶,可得
    
    $$
    \nabla\cdot\vec{B} = \vec{0},\quad 
    \frac{\partial \vec{B} }{\partial t } + \nabla\times\vec{E} = \vec{0} 
    $$    
\end{frame}

\begin{frame}
    考虑静电极限(electrostatic limit) $\vec{B}=\vec{H}=\vec{0} $,对点电荷

    $$
    \nabla\cdot\vec{D} = e\delta(\vec{r})
    $$
    
    解:
    
    $$
    \vec{D}
    =\frac{e }{4\pi r^3 } \vec{r}
    $$
    
    结合 $\vec{D},\vec{E} $ 关系和 $F=-E^2/2 $ 可得
    
    $$
    E+\frac{7 }{2 } \eta E^3
    =\frac{e }{4\gamma \pi r^2 } \left(1-\frac{\eta  }{2 } E^2 \right)^3 
    $$
    
    上式限制 $F>-1/\eta $;弱场极限 $\eta F\ll 1 $,$E(r) $ 可按 $\eta $ 展开
    
    $$
    E
    =E_{(0)} + \eta E_{(1)} + \eta^2 E_{(3)} + \mathcal{O}\left(\eta^3 \right)
    $$
    
    $E_{(1)},E_{(2)} $ 分别代表对电场 $E_{(0)} $ 的一阶和二阶修正。
\end{frame}

\begin{frame}
    比较系数可得
    
    $$
    E_{(0)}
    =\frac{e }{4\pi\gamma r^2 }
    $$
    
    $$
    E_{(1)} 
    =-\frac{7 }{2 } E_{(0)}^3 - \frac{e }{4\pi \gamma r^2 } E_{(0)}^2
    $$
    
    $$
    E_{(2)}
    =-\frac{21 }{2 } E_{(0)}^2 E_{(1)} + \frac{e }{4\pi\gamma r^2 } \left(-3E_{(0)}E_{(1)} + \frac{3 }{4 } E_{(0)}^4 \right)
    $$
    
    弱场极限下
    
    $$
    E
    \approx \frac{e }{4\pi\gamma r^2 } - 5\eta\left(\frac{e }{4\pi\gamma r^2 }  \right)^3 + \frac{273 }{4 } \eta^2\left(\frac{e }{4\pi\gamma r^2 }  \right)^5 + \mathcal{O}\left(\eta^3 \right)
    $$
    
    对于很小的 $r$ 和任意的 $\eta$,电场最大值
    
    $$
    E_{\max}
    =\sqrt{\frac{2 }{\eta } }
    $$
    
    NLE 模型中电场有限。当 $\eta\to 0 $,电场发散。
    
\end{frame}

\subsection{点电荷的能量}

\begin{frame}
    希尔伯特应力-能量张量(Hilbert stress-energy tensor)

    $$
    T_{\mu\nu}^H
    \equiv -\frac{2 }{\sqrt{-g} } \left(\frac{\partial \left(\sqrt{-g}L(F) \right) }{\partial g^{\mu\nu} }  \right)\bigg|_{g=\eta}
    $$
    
    可得
    
    $$
    T_{\mu\nu}^H
    =\eta_{\mu\nu}L(F) - L_FF_\mu^\alpha F_{\nu\alpha}
    $$
    
    电能密度
    
    $$
    \rho
    =-T_t^t
    =-L_FE^2-L(F)
    =\frac{\gamma E^2\left[1+\frac{3 }{2 } \eta E^2 \left(4+\frac{\eta }{2 } E^2 \right) \right] }{2\left(1-\frac{\eta }{2 } E^2 \right)^3 } 
    $$
    
    总电能
    
    $$
    \epsilon
    =4\pi\int_{0}^{+\infty} \rho(r)r^2\mathrm{d}r
    $$
    
    转化为对 $E $ 的积分
    
    $$
    \epsilon
    =\frac{e^{3/2} }{\sqrt{4\pi\gamma} } \int_{0}^{\sqrt{\frac{2 }{\eta  } }} \frac{\sqrt{\left(2-\eta E^2 \right)\left[4+3\eta E^2\left(8+\eta E^2 \right) \right]\left[4+\eta E^2\left(52+21\eta E^2 \right) \right]} }{16\sqrt{E}\left(2+7\eta E^2 \right)^{5/2} } \mathrm{d}E
    $$
    
    总能量有限。当 $\eta\to 0$,点电荷自能发散。
\end{frame}

\subsection{真空双折射}

\begin{frame}
        考虑平面电磁波 $(\vec{e},\vec{b}) $ 沿 $z $ 轴在两片平行电容板间传播,$x $ 轴方向有匀强电场。外电场 $\bar{\vec{E}}=(\bar{E},0,0) $,总电场 $\vec{E}=\vec{e}+\bar{\vec{E}},\vec{B}=\vec{b} $,设 $\vec{e} $ 远小于 $\bar{\vec{E}} $,拉氏量

    $$
    L\left(\vec{e}+\bar{\vec{E}},\vec{b} \right)
    =\gamma\frac{\left\{\frac{3 }{2 } \eta\left[\vec{b}^2-\left(\vec{e}+\bar{\vec{E}} \right)^2 \right] - 1 \right\}\left[\vec{b}^2-\left(\vec{e}+\bar{\vec{E}} \right)^2 \right] }{2\left\{1+\frac{\eta }{2 } \left[\vec{b}^2-\left(\vec{e} + \bar{\vec{E}}^2 \right) \right] \right\}^2 } 
    $$
    
    忽略高阶项
    
    $$
    L^{(2)}(\vec{e}+\bar{\vec{E}},\vec{b})
    =\frac{\gamma\eta\left(5+\frac{7 }{2 } \eta \bar{\vec{E}}^2 \right) }{\left(1-\frac{\eta }{2 } \bar{\vec{E}}^2 \right)^4 }\left(\vec{e}\cdot\bar{\vec{E}} \right)^2 - \frac{1 }{2 } \gamma \frac{\left(1+\frac{7 }{2 } \eta \bar{\vec{E}}^2 \right) }{\left(1-\frac{\eta }{2 } \bar{\vec{E}}^2 \right)^3 }\left(\vec{b}^2-\vec{e}^2 \right) 
    $$
    
    电位移矢量和磁场强度
    
    $$
    d_i = \frac{\partial L^{(2)} }{\partial e_i } = \left(\alpha\delta_i^j+\beta\bar{E}_i\bar{E}^j \right)e_j,\quad
    h_i
    =-\frac{\partial L^{(2)} }{\partial b_i } 
    =\alpha\delta_i^j b_j 
    $$
    其中
    $$
    \beta = \frac{2\gamma \eta\left(5+\frac{7 }{2 } \eta \bar{\vec{E}}^2 \right) }{\left(1-\frac{\eta }{2 } \bar{\vec{E}}^2 \right)^4 },\quad 
    \alpha = \gamma \frac{\left(1+\frac{7 }{2 }\eta\bar{\vec{E}}^2 \right)  }{\left(1-\frac{\eta }{2 } \bar{\vec{E}}^2 \right)^3 }
    $$
\end{frame}

\begin{frame}    
    结合关系 $d_i=\varepsilon_i^j e_j,h_i=\left(\mu^{-1} \right)_i^j b_j $
    
    $$
    \varepsilon_{ij} = \alpha\delta_{ij} + \beta\bar{E}_i\bar{E}_j ,\quad
    \left(\mu^{-1} \right)_{ij} = \alpha\delta_{ij} 
    $$
    
    平面波麦克斯韦方程
    
    $$
    k_i d^i = k_i b^i = 0,\quad
    \vec{k}\times\vec{e} = \omega\vec{b},\quad
    \vec{k}\times\vec{h} = -\omega\vec{d} 
    $$
    
    
    $$
    \left\{\varepsilon^{ijk}\varepsilon_{lmn} \left(\mu^{-1} \right)_k^l k_j k^m+\omega^2\epsilon_n^i \right\}e^n = 0
    $$
    
    $\varepsilon_{ijk} $ 是反对称张量。矩阵形式
    
    $$
    \Lambda \vec{e} = 0 
    $$
    
    $$
    \Lambda
    \equiv \begin{bmatrix}
    -k^2 \alpha+\omega^2\left(\alpha+\beta\bar{E} \right) &0 &0 \\
    0 &-k^2\alpha+\omega^2\alpha &0 \\
    0 &0 &\omega^2\alpha
    \end{bmatrix} 
    $$
    
    由 $\mathrm{det}(\Lambda)=0 $ 可知电场有两种模式。两种模式定义了色散关系。折射率定义为 $n\equiv k/\omega $,因此有两种不同的折射率
    
    $$
    n_\parallel = \sqrt{1+\frac{\beta }{\alpha } \bar{E}^2},\quad
    n_\perp = 1 
    $$
    
    不同偏振的电磁波有不同的速度 $v_\parallel=n_\parallel^{-1},v_\perp=1 $
\end{frame}

\subsection{拉格朗日量的因果性和单一性条件}

\subsection{新 NLE 与广义相对论的耦合}

\subsection{结论}

\begin{frame}
    \frametitle{\textbf{}}

\end{frame}



\section{Nonlinear Electrodynamics in f(T) Gravity and Generalized Second Law of Thermodynamics}



\subsection{非线性电动力学模型和场方程}

\subsection{点电荷的能量}

\subsection{真空双折射}

\subsection{拉格朗日量的因果性和单一性条件}

\subsection{新NLE与广义相对论的耦合}

\subsection{结论}



\section{Nonlinear electrodynamics and black holes}



\subsection{非线性电动力学模型和场方程}

\subsection{点电荷的能量}

\subsection{真空双折射}

\subsection{拉格朗日量的因果性和单一性条件}

\subsection{新NLE与广义相对论的耦合}

\subsection{结论}
    
\begin{frame}[noframenumbering]
    \centering
    {\fontsize{40}{50}\selectfont Thank You!}
\end{frame}

\end{document}