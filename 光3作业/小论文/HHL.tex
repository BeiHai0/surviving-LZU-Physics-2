%% ****** Start of file apstemplate.tex ****** %
%%
%%
%%   This file is part of the APS files in the REVTeX 4.2 distribution.
%%   Version 4.2a of REVTeX, January, 2015
%%
%%
%%   Copyright (c) 2015 The American Physical Society.
%%
%%   See the REVTeX 4 README file for restrictions and more information.
%%
%
% This is a template for producing manuscripts for use with REVTEX 4.2
% Copy this file to another name and then work on that file.
% That way, you always have this original template file to use.
%
% Group addresses by affiliation; use superscriptaddress for long
% author lists, or if there are many overlapping affiliations.
% For Phys. Rev. appearance, change preprint to twocolumn.
% Choose pra, prb, prc, prd, pre, prl, prstab, prstper, or rmp for journal
%  Add 'draft' option to mark overfull boxes with black boxes
%  Add 'showkeys' option to make keywords appear
\documentclass[aps,prl,twocolumn,groupedaddress]{revtex4-2}
%\documentclass[aps,prl,preprint,superscriptaddress]{revtex4-2}
%\documentclass[aps,prl,reprint,groupedaddress]{revtex4-2}
\usepackage{natbib}
\usepackage[export]{adjustbox}
\usepackage{graphicx}
\usepackage{float}
\usepackage{braket}
\usepackage[utf8]{inputenc}
\usepackage{ctex}
\usepackage{amsmath}

% You should use BibTeX and apsrev.bst for references
% Choosing a journal automatically selects the correct APS
% BibTeX style file (bst file), so only uncomment the line
% below if necessary.
%\bibliographystyle{apsrev4-2}

\begin{document}

% Use the \preprint command to place your local institutional report
% number in the upper righthand corner of the title page in preprint mode.
% Multiple \preprint commands are allowed.
% Use the 'preprintnumbers' class option to override journal defaults
% to display numbers if necessary
%\preprint{}

%Title of paper
\title{HHL算法及其应用}

% repeat the \author .. \affiliation  etc. as needed
% \email, \thanks, \homepage, \altaffiliation all apply to the current
% author. Explanatory text should go in the []'s, actual e-mail
% address or url should go in the {}'s for \email and \homepage.
% Please use the appropriate macro foreach each type of information

% \affiliation command applies to all authors since the last
% \affiliation command. The \affiliation command should follow the
% other information
% \affiliation can be followed by \email, \homepage, \thanks as well.
\author{林照翔}
%\email[]{Your e-mail address}
%\homepage[]{Your web page}
%\thanks{}
%\altaffiliation{}
\affiliation{2022级理论物理2班 320220935801}

%Collaboration name if desired (requires use of superscriptaddress
%option in \documentclass). \noaffiliation is required (may also be
%used with the \author command).
%\collaboration can be followed by \email, \homepage, \thanks as well.
%\collaboration{}
%\noaffiliation

\date{\today}

\begin{abstract}
% insert abstract here
\end{abstract}

% insert suggested keywords - APS authors don't need to do this
%\keywords{}

%\maketitle must follow title, authors, abstract, and keywords
\maketitle

% body of paper here - Use proper section commands
% References should be done using the \cite, \ref, and \label commands
\section{1. 引言}

HHL算法是由 Harrow, Hassidim 和 Lloyd 提出的一种求解量子线性系统的量子算法,其利用量子态的相干叠加与纠缠等特性实现稀疏线性方程组 $A\vec{x}=\vec{b}$ 的快速求解,与经典的线性方程组求解算法相比在特定情况下可以达到指数级加速。求解线性方程组是解决很多量子应用相关问题的基础,因此 HHL 算法是许多复杂量子算法的基本组成部分,且广泛应用于量子支持向量机、量子判别分析、量子线性回归、量子无监督学习、量子神经网络等量子机器学习算法中。在大数据时代,HHL 算法带来的加速收益相当可观。
% Put \label in argument of \section for cross-referencing
%\section{\label{}}

\section{2. HHL算法基本原理}

\subsection{A. 定义}

一个线性系统问题(linear system problem, LSP)可以表述为

$$
A\vec{x} = \vec{b}
$$

其中 $A $ 是  $N_b\times N_b$ 厄米矩阵,$\vec{x},\vec{b}$ 是归一化的 $N_b$ 维矢量,$N_b=2^{n_b}$。$A$ 和 $\vec{b}$ 是已知的,而 $\vec{x}$ 待求解。

HHL 算法包含五个主要组成部分:初态制备、量子相位估计(quantum phase estimation, QPE)、辅助量子比特旋转、逆量子相位估计(inverse quantum phase estimation, IQPE) 和测量。

HHL 算法中,$\vec{b},\vec{x} $ 的 $N_b$ 个分量编码为 $n_b$ 个量子比特的振幅。这些量子比特称为 b-寄存器 $\ket{}_b$。

除此之外还有 c-寄存器 $\ket{}_c$,用于储存矩阵 $A$ 在量子相位估计后得到的本征值。

最后一部分是辅助量子比特 $\ket{}_a$。

设厄米矩阵 $A$ 的本征态和本征值分别为 $\left\{\ket{u_i} \right\}$ 和 $\left\{\lambda_i \right\}$,则 $A$ 可进行谱分解

$$
A = \sum_{i=0}^{2^{n_b}-1} \lambda_i \ket{u_i}\bra{u_i}
$$

$A$ 是对角的,因此容易得到 $A^{-1}$

$$
A^{-1} = \sum_{i=0}^{2^{n_b}-1} \lambda_i^{-1} \ket{u_i}\bra{u_i}
$$

$\ket{b}$ 可以表达为

$$
\ket{b} = \sum_{j=0}^{2^{n_b}-1} b_j \ket{u_j}
$$

因此,由方程 $A\ket{x}=\ket{b}$ 有

$$
\ket{x} = A^{-1}\ket{b} = \sum_{i=0}^{2^{n_b}-1} \lambda_i^{-1}b_i \ket{u_i}
$$

由于 $\ket{x},\ket{b}$ 是归一的,因此有

$$
\sum_{j=0}^{2^{n_b}-1}\left|b_j \right|^2 = 1
$$

$$
\sum_{i=0}^{2^{n_b}-1} \left|\lambda_i^{-1}b_i\right|^2 = 1
$$

\subsection{B. 初态制备}

共有 $n_b+n+1$ 个量子比特,初始化为

$$
\ket{\Psi_0}
=\ket{0\cdots 0}_b\ket{0\cdots 0}_c\ket{0}_a
=\ket{0}^{\otimes n_b}\ket{0}^{\otimes n}\ket{0}
$$

设 $\vec{b} = \left(\beta_0,\beta_1,\cdots,\beta_{N_b-1} \right)^{\mathrm{T}} $则 b-寄存器 $\ket{0\cdots 0}_b$ 需要制备为

$$
\ket{b}
=\beta_0\ket{0}+\beta_1\ket{1}+\cdots\beta_{N_b-1}\ket{N_b-1}
$$

总态变为

$$
\ket{\Psi_1}
=\ket{b}_b\ket{0\cdots 0}_c\ket{0}_a
$$

为了方便,在不引起歧义的情况下,下文中 ket 的下标省略。

\subsection{C. 量子相位估计}

首先,对c-量子比特施加 Hadamard 门,得到

$$
\ket{\Psi_2}
=I^{\otimes n_b}\otimes H^{\otimes n}\otimes I\ket{\Psi_1}=
\ket{b}\frac{1 }{2^{n/2} } (\ket{0}+\ket{1})^{\otimes n} \ket{0}
$$

接着,以 c-量子比特为控制比特,依次对 $\ket{b}$ 施加受控 $U^{2^r}$ 门,其中 $r$ 为 c-量子比特的下标,$U=\mathrm{e}^{\mathrm{i}At}$。

假设 $\ket{b}$ 为 $U$ 的本征态,对应本征值为 $\mathrm{e}^{\mathrm{i}2\pi \phi}$,它们本征方程 $U$ 的本征方程

$$
U\ket{b} = \mathrm{e}^{\mathrm{i}2\pi\phi} \ket{b}
$$

当控制比特为 $\ket{0}$,$\ket{b}$ 不受影响;当控制比特为 $\ket{1}$,$U$ 将作用于 $\ket{b}$。结合本征方程可知,总态变为

$$
\begin{aligned}
\ket{\Psi_3}
&=\ket{b}\otimes \bigg[\frac{1 }{2^{n/2} } \left(\ket{0}+\mathrm{e}^{\mathrm{i}2\pi \phi 2^{n-1}}\ket{1} \right)\otimes \\
&\left(\ket{0}+\mathrm{e}^{\mathrm{i}2\pi \phi 2^{n-2}}\ket{1} \right)\otimes \cdots \otimes \left(\ket{0}+\mathrm{e}^{\mathrm{i}2\pi \phi 2^0}\ket{1} \right) \bigg]\otimes\ket{0}_a \\
&=\ket{b} \otimes \left(\frac{1 }{2^{n/2} } \sum_{k=0}^{2^n-1} \mathrm{e}^{2\pi\mathrm{i}\phi k} \ket{k} \right)\otimes \ket{0}_a
\end{aligned}
$$

然后对 c-量子比特施加逆量子傅里叶变换(IQFT),总态变为

$$
\begin{aligned}
\ket{\Psi_4}
&=\ket{b}\mathrm{IQFT}\left(\frac{1 }{2^{n/2} } \sum_{k=0}^{2^n-1} \mathrm{e}^{2\pi\mathrm{i}\phi k} \ket{k} \right) \ket{0}_a \\
&=\ket{b}\frac{1 }{2^{n/2} } \sum_{k=0}^{2^n-1} \mathrm{e}^{2\pi\mathrm{i}\phi k} \mathrm{IQFT}(\ket{k}) \ket{0}_a \\
&=\ket{b}\frac{1 }{2^{n/2} } \sum_{k=0}^{2^n-1} \mathrm{e}^{2\pi\mathrm{i}\phi k} \left(\frac{1 }{2^{n/2} } \sum_{y=0}^{2^n-1} \mathrm{e}^{-\mathrm{i}2\pi yk/N} \ket{y} \right) \ket{0}_a \\
&=\frac{1 }{2^n } \ket{b} \sum_{y=0}^{2^n-1}\sum_{k=0}^{2^n-1}\mathrm{e}^{\mathrm{i}2\pi k(\phi-y/N)}\ket{y}\ket{0}_a
\end{aligned}
$$

由上式可见,只有当 $\ket{y}$ 满足条件 $\phi-y/N=0$ 才贡献有限振幅 $\displaystyle{\sum_{k=0}^{2^n-1} \mathrm{e}^0 = 2^n }$,其他情况下贡献的振幅为零,即 $\displaystyle{\sum_{k=0}^{2^n-1}\mathrm{e}^{\mathrm{i}2\pi k (\phi-y/N)}=0,\phi-y/N\ne 0 }$,因此 $\ket{\Psi_4}$ 可以写为

$$
\ket{\psi_4}
=\frac{1 }{2^n } \ket{b} \sum_{k=0}^{2^n-1} \mathrm{e}^{\mathrm{i}2\pi k\cdot 0} \ket{N\phi}\ket{0}_a
=\ket{b}\ket{N\phi}\ket{0}_a
$$

因此,c-量子比特就储存了 $U$ 的本质值 $\mathrm{e}^{\mathrm{i}2\pi \phi} $ 的相位 $\phi$,精确度取决于用于储存的量子比特的数量 $n$。

由于 $U$ 与 $A$ 的关系为 $U = \mathrm{e}^{\mathrm{i}At}$,若 $\ket{b}=\ket{u_j}$ 则

$$
U\ket{b} = \mathrm{e}^{\mathrm{i}\lambda_j t} \ket{b}
$$

与 $U\ket{b}=\mathrm{e}^{\mathrm{i}2\pi \phi}\ket{b}$ 对比,得到 $\phi=\lambda_j t/2\pi$,于是

$$
\ket{\Psi_4}
=\ket{u_j} \ket{N\lambda_j t/2\pi}\ket{0}_a
$$

因此 $A$ 的本征值就编码到了 c-量子比特上。上面的推导均在 $\ket{b}$ 为 $A$ 的某一本征态的假设下进行。一般地,$\ket{b}$ 为 $A$ 的所有本征态的线性叠加,则

$$
\ket{\Psi_4}
=\sum_{j=0}^{2^{n_b}-1} b_j\ket{u_j}\ket{N\lambda_j t/2\pi} \ket{0}_a
$$

通常 $\lambda_j$ 不是整数。选择 $t$ 使得 $\tilde{\lambda}_j\equiv N\lambda_j t/2\pi$ 为整数,则 $\ket{\Psi_4} $ 可写为

$$
\ket{\Psi_4}
=\sum_{j=0}^{2^{n_b}-1} b_j \ket{u_j} \ket{\tilde{\lambda}_j}\ket{0}_a
$$

\subsection{D. 受控旋转和辅助比特的测量}



\section{3. HHL算法的简单应用}

% If in two-column mode, this environment will change to single-column
% format so that long equations can be displayed. Use
% sparingly.
%\begin{widetext}
% put long equation here
%\end{widetext}

% figures should be put into the text as floats.
% Use the graphics or graphicx packages (distributed with LaTeX2e)
% and the \includegraphics macro defined in those packages.
% See the LaTeX Graphics Companion by Michel Goosens, Sebastian Rahtz,
% and Frank Mittelbach for instance.
%
% Here is an example of the general form of a figure:
% Fill in the caption in the braces of the \caption{} command. Put the label
% that you will use with \ref{} command in the braces of the \label{} command.
% Use the figure* environment if the figure should span across the
% entire page. There is no need to do explicit centering.

% \begin{figure}
% \includegraphics{}%
% \caption{\label{}}
% \end{figure}

% Surround figure environment with turnpage environment for landscape
% figure
% \begin{turnpage}
% \begin{figure}
% \includegraphics{}%
% \caption{\label{}}
% \end{figure}
% \end{turnpage}

% tables should appear as floats within the text
%
% Here is an example of the general form of a table:
% Fill in the caption in the braces of the \caption{} command. Put the label
% that you will use with \ref{} command in the braces of the \label{} command.
% Insert the column specifiers (l, r, c, d, etc.) in the empty braces of the
% \begin{tabular}{} command.
% The ruledtabular enviroment adds doubled rules to table and sets a
% reasonable default table settings.
% Use the table* environment to get a full-width table in two-column
% Add \usepackage{longtable} and the longtable (or longtable*}
% environment for nicely formatted long tables. Or use the the [H]
% placement option to break a long table (with less control than 
% in longtable).
% \begin{table}%[H] add [H] placement to break table across pages
% \caption{\label{}}
% \begin{ruledtabular}
% \begin{tabular}{}
% Lines of table here ending with \\
% \end{tabular}
% \end{ruledtabular}
% \end{table}

% Surround table environment with turnpage environment for landscape
% table
% \begin{turnpage}
% \begin{table}
% \caption{\label{}}
% \begin{ruledtabular}
% \begin{tabular}{}
% \end{tabular}
% \end{ruledtabular}
% \end{table}
% \end{turnpage}

% Specify following sections are appendices. Use \appendix* if there
% only one appendix.
%\appendix
%\section{}

% If you have acknowledgments, this puts in the proper section head.
%\begin{acknowledgments}
% put your acknowledgments here.
%\end{acknowledgments}

% Create the reference section using BibTeX:
\bibliography{basename of .bib file}

\end{document}
%
% ****** End of file apstemplate.tex ******

